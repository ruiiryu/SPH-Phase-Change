\documentclass[10pt, twocolumn] {IEEEtran}
\usepackage{verbatim}
\usepackage{graphicx}

\newcommand{\myscale}{0.65}
\renewcommand{\baselinestretch}{0.960}

\begin{document}
\title{Title}
\author{Nutcha Chayanurak}
\maketitle
\begin{abstract} 
 \end{abstract}
\section{Introduction}
Heat transfer process is important for phase transition, such as ice melting and water freezing. Exemple, ice exchange heat with air surroundings. Surface of ice melt into a thin layer of water  gradually and enclose on ice when temperature of ice is equal melting point and heat of ice is equal latent heat. It same in water freezing. In this paper, We propose a new approach based on
a particle-based model for simulating ice melting and water freezing.
\section{Previous Work}
This section reviews previous concept to simulate ice-water interface for thermodynamic heat transfer process in freezing and melting phenomena. 

Kim et al.\cite{kim2004hybrid} presented a hybrid algorithm for generating a thin ice crystal growth by the thermodynamics in freezing process, such as snowflake. Although this method cannot be applied on a volume of ice, such as icicles, and heat transfer to conduct heat from water vapor to the crystal. They also proposed a method for solving icicle formation by the thin-film Stefan problem for model a thin film of water\cite{kim2006modeling}. Madrazo et al. integrated air bubbles into ice\cite{madrazo2009air}. Their method simulated a dissolved air exhausted from water-ice interface to create geometrical representations of ice with bubbles.The interface moving is determined by a particle level set method, using the velocity field calculated by Stefan problem. However this paper studied only ice growth in bulk water case. Miao and Xiao\cite{Miao:2015:PIF:2817675.2817676} presented particle-based ice freezing model to simulate the fast freezing phenomenon occurs in flowing water. This method used surface particles of fluid to exchange heat with cold sources for various heat transfer directions can be controlled flexibly.
 
 Iwasaki et al.\cite{iwasaki2010fast} proposed a particle-based simulation method for the heat transfer in melting process. They improved a simple method for the interfacial forces between water-ice particle to simulate the flow of meltwater on ice and the formation of water droplets. Although this method can be applied to the simulation of water freezing, it was not further studied.
 Domaradzki and Martyn\cite{domaradzki2014improved} proposed new particle-based air model represents the natural air convection on heat transfer process  which allows for considering  local changes in air temperature and a new external heat source model controls direction of melting.
 Lii and Wong\cite{lii2014ice} proposed new attribute called virtual water for ice particles. Their approach also handles the transition between the virtual water on ice surface and the water particles and computes the density field defined by the ice particles and the virtual water.

\section{Method}
\section{Result}
\section{Conclusion}
\section{Acknowledgement}

\nocite{*}
\bibliographystyle{unsrt}
\bibliography{BibFile}

\end{document}




